\documentclass[12pt]{article}

% Packages
\usepackage{graphicx}    % For including figures
\usepackage{amsmath}     % For math symbols
\usepackage{amssymb}     % For more math symbols
\usepackage{geometry}    % For page dimensions
\usepackage{hyperref}    % For hyperlinks
\usepackage{caption}     % For figure captions
\usepackage{cite}        % For citations
\usepackage{fancyhdr}    % For headers and footers

% Page settings
\geometry{
    a4paper,
    left=25mm,
    right=25mm,
    top=30mm,
    bottom=30mm
}

% Custom header and footer
\pagestyle{fancy}
\fancyhead[L]{Wirtz Pumpe} % Left side of the header
\fancyhead[R]{Kian Moldenhauer, Alexander Schulzki}            % Right side of the header
\fancyfoot[C]{\thepage}             % Centered page number in the footer

\begin{document}

% Title section
\title{Basic Layout Wirtz Pumpe}
\author{Kian Moldenhauer & Alexander Schulzki\\Immanuel-Kant-Gymnasium\\\texttt{kian.moldenhauer@gmail.com}}
\date{October 8, 2024}
\maketitle

% Abstract
\begin{abstract}
Diese Arbeit untersucht die Performance und Effizienz der Wirtz-Pumpe unter verschiedenen Betriebsbedingungen. In einer Reihe von Experimenten wurden Parameter wie Durchflussrate, Förderhöhe und Energieverbrauch analysiert, um den Wirkungsgrad der Pumpe zu bewerten. Die Versuchsanordnung bestand aus einem geschlossenen Wasserkreislauf, in dem die Pumpe Wasser über unterschiedliche Zeitintervalle förderte.

Die Ergebnisse zeigen, dass die Wirtz-Pumpe bei niedrigen Förderhöhen die höchste Effizienz erreicht, während der Energieverbrauch bei zunehmender Förderhöhe ansteigt. Durch die Berechnung des Wirkungsgrads und den Vergleich von geförderter Wassermenge und verbrauchter Energie wurde das Verhalten der Pumpe unter praxisnahen Bedingungen untersucht. Diese Arbeit liefert wichtige Erkenntnisse zur Optimierung der Pumpe und ihrer Einsatzmöglichkeiten, insbesondere in Anwendungen, die eine hohe Effizienz bei geringem Energieverbrauch erfordern.\end{abstract}

\tableofcontents  % Generates the table of contents
\newpage          % Start a new page for the introduction

\section{Introduction}
DDie Wirtz-Pumpe ist ein innovatives hydraulisches Gerät, das für die effiziente Förderung von Flüssigkeiten entwickelt wurde. Sie basiert auf einem spiralförmigen Design, das die kinetische Energie des Wassers optimal nutzt, um eine gleichmäßige und kontinuierliche Strömung zu gewährleisten. Diese Art von Pumpe hat in der Vergangenheit vor allem in landwirtschaftlichen Anwendungen, in Brunnen- und Bewässerungssystemen, Beachtung gefunden, da sie ohne komplexe Mechanik oder hohe Energiezufuhr arbeitet. Ihre Effizienz hängt stark von der Förderhöhe und der Viskosität der Flüssigkeit ab, was sie ideal für spezifische Anwendungsbereiche macht.

Obwohl zahlreiche Studien die Funktion und Bauweise herkömmlicher Pumpensysteme untersucht haben, gibt es weniger Forschung zur Performance und Effizienz der Wirtz-Pumpe unter verschiedenen Bedingungen. Dieses Forschungsvorhaben zielt darauf ab, die Leistungsfähigkeit der Pumpe unter variierenden Betriebsparametern wie Förderhöhe, Durchflussrate und Energieverbrauch zu analysieren. Durch diese Untersuchung soll ein besseres Verständnis der Effizienzmechanismen der Wirtz-Pumpe gewonnen und potenzielle Einsatzmöglichkeiten in verschiedenen Bereichen identifiziert werden. Die Ergebnisse sollen Aufschluss darüber geben, wie die Pumpe unter realistischen Bedingungen optimiert werden kann.
% Methods
\section{Methods}
In dieser Studie wurde die Performance und Effizienz der Wirtz-Pumpe in kontrollierten Laborversuchen untersucht. Die verwendeten Materialien umfassten eine handelsübliche Wirtz-Pumpe, einen Wasserkreislauf mit variabler Förderhöhe, ein Durchflussmessgerät, ein Energieverbrauchsmessgerät und ein Wasserreservoir.

Materialien:
\begin{enumerate}
\item Wirtz-Pumpe
\item Durchflussmessgerät
\item Energieverbrauchsmessgerät
\item Variabler Wasserkreislauf
\item Messzylinder zur Volumenmessung
\item Stoppuhr
\end{enumerate}

Vorgehensweise:
\begin{itemize}
\item Versuchsaufbau: Ein geschlossener Wasserkreislauf wurde eingerichtet, in dem die Wirtz-Pumpe Wasser aus einem Reservoir förderte. Die Förderhöhe wurde systematisch in mehreren Stufen (z.B. 1 m, 2 m, 3 m) angepasst.

\item Messung der Durchflussrate: Das Wasservolumen, das über einen definierten Zeitraum gefördert wurde, wurde mit einem Durchflussmessgerät gemessen. Diese Messungen wurden für jede Förderhöhe wiederholt.

\item Energieverbrauch: Der elektrische Energieverbrauch der Pumpe wurde für jede Förderhöhe mit einem Energieverbrauchsmessgerät erfasst.

\item Berechnung der Effizienz: Der Wirkungsgrad der Pumpe wurde durch den Vergleich der gepumpten Wassermenge mit dem verbrauchten Energieaufwand ermittelt.
\end{itemize}
%formatieren
Ansatz:
Die Experimente wurden wiederholt, um konsistente und reproduzierbare Ergebnisse zu gewährleisten. Die erhaltenen Daten wurden analysiert, um die Effizienz der Pumpe unter verschiedenen Bedingungen zu bewerten und Rückschlüsse auf die optimale Betriebsweise zu ziehen.
% Results
\section{Ergebnisse}
Ergebnisse representation und auffassung
% Example of a figure
\begin{figure}[h]
    \centering
    \includegraphics[width=0.6\textwidth]{example-image} % Replace with your image file
    \caption{Beispielbild}
    \label{fig:example}
\end{figure}

% Discussion
\section{Diskussion}
Interpretieren der Ergebnisse und geben von Erklärungen für beobachtete Trends. Die Vergleichen Ergebnisse mit früheren Studien.

% Conclusion
\section{Zusammenfassung}
zusammenFassen der Forschungsergebnisse und heben die wichtigsten Ergebnisse und deren Bedeutung hervor. 
% References
\begin{thebibliography}{9}
\bibitem{ref1} Author Name, \textit{Title of the Reference}, Journal Name, Year.
\bibitem{ref2} Author Name, \textit{Title of the Reference}, Journal Name, Year.
\end{thebibliography}

\end{document}
