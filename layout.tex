\documentclass[12pt]{article}

% Packages
\usepackage{graphicx}    % For including figures
\usepackage{amsmath}     % For math symbols
\usepackage{amssymb}     % For more math symbols
\usepackage{geometry}    % For page dimensions
\usepackage{hyperref}    % For hyperlinks
\usepackage{caption}     % For figure captions
\usepackage{cite}        % For citations
\usepackage{fancyhdr}    % For headers and footers

% Page settings
\geometry{
    a4paper,
    left=25mm,
    right=25mm,
    top=30mm,
    bottom=30mm
}

% Custom header and footer
\pagestyle{fancy}
\fancyhead[L]{Research Paper Title} % Left side of the header
\fancyhead[R]{Your Name}            % Right side of the header
\fancyfoot[C]{\thepage}             % Centered page number in the footer

\begin{document}

% Title section
\title{Basic Layout of a Research Paper in \LaTeX}
\author{Your Name\\Your Institution\\\texttt{your.email@example.com}}
\date{\today}
\maketitle

% Abstract
\begin{abstract}
This is a concise summary of your research paper, usually around 150–250 words. It should briefly describe the purpose, methodology, key results, and conclusions of your research.
\end{abstract}

\tableofcontents  % Generates the table of contents
\newpage          % Start a new page for the introduction

% Introduction
\section{Introduction}
This section provides background information and sets the context for your research. It should explain the problem, highlight relevant literature, and specify the research objectives.

% Methods
\section{Methods}
Describe the methodology used to conduct the research. Include details of materials, procedures, and the overall approach.

% Results
\section{Results}
Present the key findings of the study. Use tables, figures, and descriptive text to illustrate your results. 

% Example of a figure
\begin{figure}[h]
    \centering
    \includegraphics[width=0.6\textwidth]{example-image} % Replace with your image file
    \caption{This is an example of a figure.}
    \label{fig:example}
\end{figure}

% Discussion
\section{Discussion}
Interpret the results and provide explanations for observed trends. Compare the findings with previous studies.

% Conclusion
\section{Conclusion}
Summarize the research, highlighting the main findings and their significance. Suggest areas for future research.

% References
\begin{thebibliography}{9}
\bibitem{ref1} Author Name, \textit{Title of the Reference}, Journal Name, Year.
\bibitem{ref2} Author Name, \textit{Title of the Reference}, Journal Name, Year.
\end{thebibliography}

\end{document}
